%\pagestyle{mystyle}
\chapter{The Strong Interaction at zero and non-zero temperature}
This chapter is devoted to a theoretical overview of Quantum
Chromo-Dynamics (QCD) and its key features within the Euclidean Path Integral 
formulation, as well as the definition of the theory 
at non-zero values of the temperature in thermal equilibrium.
\section{QCD in the vacuum}
The strong nuclear interaction is described by the sector of the Standard Model
known as Quantum Chromo-Dynamics. With just a few free parameters, it is 
able to account for the experimentally observed properties of hadrons from 
low to high energy scales (see \cite{QCD50} for a recent review celebrating 50
years of the theory). To understand its formal definition, it is 
instructive to consider the most general classical Euclidean Lagrangian density one can write requiring 
$\mathrm{SO}(4)$ invariance and renormalizability - in 4 dimensions - for a theory with a number $N_f$ 
of spin $\frac{1}{2}$ fields transforming in the fundamental representation of 
a unitary gauge group $\mathrm{SU}(N_c)$\footnote{For the notation 
and conventions used in this thesis, see Appendix \ref{app:A}}, which reads\footnote{The additional
dimension 4, $CP$-violating operator that can be constructed from the gauge fields will be 
omitted throughout this thesis, due to the extremely small value of the
QCD theta-term observed from experiment (see \cite{theta\_rev} for a review)}
\begin{equation}
	\mathcal{L}_\mathrm{QCD}\lsb x;A_\mu,\psi,\bar{\psi}\rsb =\frac{1}{2g_0^2}\Tr \lcb F_{\mu\nu}(x) F_{\mu\nu}(x) \rcb + \sum_{j=1}^{N_f}
	\bar{\psi}_j(x)\lp \gamma_\mu D_\mu + m_j \rp \psi_j(x).
	\label{eq:gen_lag}
\end{equation} 
The first term in the above equation is the Yang-Mills action for the non-Abelian
gauge fields $A_\mu$ with bare coupling constant $g_0^2$, while the second term is the gauge invariant Lagrangian
for each of the $N_f$ Dirac spinors $\psi_j,\ \bar{\psi}_j$ with bare masses $m_j$.
While it will be instrumental to keep $N_c$ - the number of \say{colors} - and 
$N_f$ - the number of quark \say{flavours} - general for parts of this section, 
QCD is defined by considering $N_c=3$ and $N_f=6$ with no
degenerate quark masses. With a slight abuse of notation, I will refer to both
this case and to the general case as QCD.
Integrating the Lagrangian density over spacetime yields the QCD action:
\begin{equation}
	\mathcal{S}_\mathrm{QCD}\lsb A_\mu,\psi,\bar{\psi} \rsb = 
	\int d^4x\ \mathcal{L}_\mathrm{QCD}\lsb x; A_\mu, \psi, \bar{\psi}\rsb.
	\label{eq:qcd_action}
\end{equation}
The promotion of a classical theory with a given action
to a Quantum Field Theory can be achieved within the Path Integral formalism;
the central object in this formalism is the system's partition function which,
inspired by its analogue in classical statistical mechanics, takes into account
all the system's possible microstates - the values of the fields at each spacetime
point, or \say{configurations} - each weighted with a suitable factor. Postponing
the rigorous definition and meaning of the integration measure over fields to
section \ref{sec:LQCD},
the formal expression of the QCD partition function is
\begin{equation}
	\mathcal{Z}_\mathrm{QCD}=\int\mathfrak{D}\lsb\text{fields}\rsb\ 
\exp{-\mathcal{S}_\mathrm{QCD}\lsb A_\mu,\psi,\bar{\psi} \rsb -\mathcal{S}_\mathrm{GF}\lsb A_\mu\rsb
-\mathcal{S}_\mathrm{FP}\lsb C,\bar{C}\rsb
},\label{eq:contZ} 
\end{equation}
where in addition to the Boltzmann weight associated to the QCD action, two extra 
terms have been introduced to account for the integration over physically equivalent
configurations of gauge fields in equation \eqref{eq:contZ}. 
The first is the gauge-fixing term, which needs to be introduced to render 
the kernel of the Yang-Mills action invertible (and thus
be able to define a propagator for vector fields). In the covariant Lorenz gauge,
it reads
\begin{equation}
	\mathcal{S}_\mathrm{GF}\lsb A_\mu\rsb=-\frac{1}{2\alpha}\int d^4x\ 
\lp\partial_\mu A_\mu(x) \rp^2,\label{eq:gfact}
\end{equation}
where $\alpha$ is an arbitrary gauge parameter. The addition of this term
explicitly breaks gauge invariance at the level of the action, though it is
 retained when computing physical observables.
The second term, which arises during the gauge fixing procedure in non-abelian
Yang-Mills theories, is the so-called 
Faddeev-Popov action for the auxiliary ghost fields $C$ and $\bar{C}$, which reads
\begin{equation}
	\mathcal{S}_\mathrm{FP}\lsb C,\bar{C} \rsb = \frac{2}{g_0^2}\int d^4x
\Tr\lcb\partial_\mu\bar{C}(x) D_\mu C(x)\rcb.\label{eq:fpact}
\end{equation}
This is the action of a complex scalar field transforming in the adjoint representation
of the gauge group; however, ghost fields are described by anticommuting 
Grassman variables, and they therefore violate the spin-statistics theorem.
This means that they do not create or annihilate physical states, but they need
to be taken into account in computations to obtain correct results 
\footnote{For instance, in a pertubative expansion of 
equation \eqref{eq:contZ}, ghost propagators would only enter as
internal legs of Feynman diagrams.}.\\
Physical observables such as particle masses, decay widths and cross sections 
are encoded in correlation functions of local fields, which take the form
of statistical expectation values:
\begin{equation}
	\la O\lp x_1,\dots,x_n\rp\ra\equiv\frac{1}{\mathcal{Z}_\mathrm{QCD}}
\int\mathfrak{D}\lsb\mathrm{fields}\rsb\ O\lp x_1,\dots,x_n\rp\exp{-\mathcal{S}_\mathrm{cont}[\text{fields}]},\label{eq:expv}
\end{equation}
where $\displaystyle O\lp x_1,\dots, x_n \rp$ denotes a general operator 
constructed from the fields appearing in the action of equation \eqref{eq:contZ}
with support at spacetime points $\displaystyle \lcb x_1,\dots,x_n \rcb$, and 
the continuum action is defined as
$\mathcal{S}_\mathrm{cont}\equiv\mathcal{S}_\mathrm{QCD}+\mathcal{S}_\mathrm{GF}+\mathcal{S}_\mathrm{FP}$. 
To conclude this section, I remark that the definition of the path integral can
equivalently be performed (and it historically has been) considering the usual 
real-time Minkowski metric, with fields being operators acting on the theory's 
Hilbert space. The key observation is that there exists a one-to-one correspondence 
between vacuum expectation values of time-ordered products of operators in Minkowski 
time and expectation values of fields in imaginary Euclidean time:
\begin{equation}
	\la O_1(x_1^\mathrm{E})\dots O_n(x_n^\mathrm{E})\ra \longleftrightarrow 
	\bra{\Omega}\mathcal{T}\lcb\hat{O}_1(x_1^\mathrm{M})\dots\hat{O}_n(x_n^\mathrm{M})\rcb \ket{\Omega},\label{eq:euc_min_map}
\end{equation}
where $\ket{\Omega}$ is the theory's vacuum state, $\mathcal{T}$ denotes the 
time-ordering operation and hats indicate operators. Finally, the usual relation
between Euclidean and Minkowski coordinates is
\begin{equation}
	x^\mathrm{E}_0\equiv i \lp x^\mathrm{M}\rp^0,\ \ x^\mathrm{E}_j\equiv \lp x^\mathrm{M}\rp^j.\label{eq:euclidean_x}
\end{equation}
Equations \eqref{eq:euc_min_map} and \eqref{eq:euclidean_x} imply that all the information
contained in a real-time Minkowskian correlation function can in principle be reconstructed
from an Euclidean expectation value by performing an analytic continuation from 
imaginary to real time. Unless otherwise specified, all coordinate and field 
values are assumed to be Euclidean in this thesis.
\subsection{Renormalization}

Discuss the symmetries
of QCD and their breaking pattern, the renormalization procedure ($\Rightarrow$
asymptotic freedom) and their implications on the hadron spectrum. Discuss the need
for non-perturbative methods in $T=0$ QCD.\\
\section{Thermal QCD}
Proceed with the discussion of QFT at finite temperature: physical motivation
(cosmology, heavy ions) and formal description: ensembles, partition function,
compactification of euclidean time, Matsubara frequencies, shifted boundary 
conditions. Discuss the IR problem and the EFT approach to it ($\rm QCD_3$). 
Discuss the need for non-perturbative methods in $T\neq 0$ QCD.
