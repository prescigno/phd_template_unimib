\documentclass{report}
\usepackage{graphicx}
\usepackage{geometry}
\usepackage{caption}
\usepackage{subcaption}
\usepackage{amssymb}
\usepackage{amsmath}
\usepackage{enumitem}
\usepackage{physics}
\usepackage{hyperref}
\usepackage{cleveref}
\usepackage{url}
\usepackage{dirtytalk}
\usepackage{color}
\usepackage{calc}
\usepackage{float}
\usepackage{epsfig}
\usepackage{dsfont}
\usepackage{lipsum}
\usepackage{tikz-feynman}
\usepackage{multirow}
\usepackage{makecell}
\usepackage{MnSymbol}%\sumint
\usepackage{wrapfig}
\usepackage{textpos}
\usepackage{biblatex}
\usepackage[integrals]{wasysym}
\addbibresource{bibliography/bibliography.bib}
\usepackage{fancyhdr}
%custom page style for report class
\fancypagestyle{mystyle}{
\pagestyle{fancy}
\fancyhf{} % clear all headers      
\fancyhead[R]{\rmfamily \small \nouppercase \rightmark}
\fancyhead[L]{\rmfamily \small \nouppercase \leftmark}  
\fancyfoot[C]{\thepage}
\renewcommand{\headrulewidth}{0pt}
\renewcommand{\footrulewidth}{0pt}}
\newcommand{\alert}[1]{{\color{red} #1}}


% macros for fast scaling parentheses
\newcommand{\lp}{\left(}
\newcommand{\rp}{\right)}

\newcommand{\lsb}{\left[}
\newcommand{\rsb}{\right]}

\newcommand{\lcb}{\left\{}
\newcommand{\rcb}{\right\}}

\newcommand{\la}{\left\langle}
\newcommand{\ra}{\right\rangle}
\definecolor{red}{rgb}{0.631, 0.094, 0.094}
\begin{document}

%%% TITLE PAGE %%%

\thispagestyle{empty}
\begin{textblock*}{\textwidth}(-2cm,-2.5cm)
\includegraphics[height=0.1\paperheight]{logos/Bicocca_transparent.png}	
\end{textblock*}
\begin{textblock*}{\textwidth}(1.9cm,-2.6cm)
\begin{flushleft}
{\color{red}\Large{SCUOLA DI DOTTORATO}}\\[2mm]
{\color{red}\large{UNIVERSIT\`A DEGLI STUDI DI MILANO-BICOCCA}}
\end{flushleft}
\end{textblock*}
\vspace{2cm}
\centerline{\Large{Department of Physics \say{G. Occhialini}}}
\vspace{0.5cm}
\centerline{\Large{Ph. D. program in Physics and Astronomy, XVIII cycle}}
\vspace{2mm}
\centerline{\Large{Curriculum of Theoretical Physics}}
\vspace{2cm}
\centerline{\huge{ The non-static screening spectrum}}
\centerline{\huge{of high-temperature Quantum Chromo-Dynamics}}
\vspace{2cm}
\centerline{\LARGE{Pietro Rescigno}}
\vspace{2mm}
\centerline{\Large{Registration number: 826333}}
\vspace{2cm}
\begin{flushleft}
\LARGE{Tutor: Leonardo Giusti}\\[1.5mm]
\LARGE{Supervisor: Marco Cè}\\[3cm]
\LARGE{Coordinator: Stefano Ragazzi}
\end{flushleft}
\vfill

\centerline{\LARGE{Academic Year 2025/2026}}
%%%%%%%%%%%%%%%%%
%%% ABSTRACT (Techincal)%%%%
\begin{abstract}
   A one-page \say{technical} abstract that summarizes
   the research work, similar to the one in the 
   paper on baryonic screening masses.	
\end{abstract}
%%% PUBLICATIONS %%%
\pagenumbering{roman}

{\centering
{\Large\textbf{Scientific Output}}\\[5mm]
\centerline{Here I list the works I have published and 
my contributions to}
\centerline{international conferences, either
as a speaker or as an author.}
\vspace{3mm}
\centerline{Bibliometric information can be retrieved on my}
\centerline{\href{https://inspirehep.net/authors/2727845?ui-citation-summary=true}{iNSPIRE} or \href{https://scholar.google.com/citations?user=3bmCuiAAAAAJ&hl=it}{Google Scholar} pages.}}


% Name & headline

%\switchcolumn
\section*{List of publications}
$\bullet$ L. Giusti, T. Harris, D. Laudicina, M. Pepe, PR, \textit{Baryonic screening masses in QCD at high temperature}, Phys. Lett. B 855 (2024) 138799,
\href{https://doi.org/10.1016/j.physletb.2024.138799}{https://doi.org/10.1016/j.physletb.2024.138799}, \href{https://arxiv.org/abs/2405.04182}{arxiv:2405.04182}\\[5mm]
$\bullet$ L. Giusti, M. Laine, D. Laudicina, M. Pepe, PR, \textit{Baryonic thermal screening mass at NLO}, JHEP 06 (2024) 205, \href{https://doi.org/10.1007/JHEP06(2024)205}{https://doi.org/10.1007/JHEP06(2024)205}, \href{https://arxiv.org/abs/2405.03975}{arxiv:2405.03975}

\section*{Contributions to conferences}
As a \textbf{speaker:}
\begin{itemize}
\item PR, L. Giusti, T. Harris, M. Laine, D. Laudicina, M. Pepe, \textit{Nucleon screening 
masses in QCD at high temperature}, talk - Strong and
Electroweak Matter 2024, Goethe Universit\"at Frankfurt, DE - \href{https://indico.physik.uni-bielefeld.de/event/100/contributions/226/}{Indico page} \\
\item PR, L. Giusti, T. Harris, D. Laudicina, M. Pepe, \textit{Baryonic screening masses in 
high temperature QCD}, talk -
The 40$^{\rm th}$ International Symposium on Lattice Field Theory, Fermilab, USA - \href{https://pos.sissa.it/453/196/}{PoS(LATTICE2023)}
\end{itemize}
As an \textbf{author:}
\begin{itemize}
    \item L. Giusti, M. Bresciani, M. Dalla Brida, T. Harris, D. Laudicina, M. Pepe,
    PR, \textit{Non-perturbative thermal QCD at very high temperatures: computational strategy and hadronic screening masses}, talk - The 41$^{\rm st}$ International Symposium on Lattice Field Theory, University of Liverpool, UK - to appear in 
    \textit{Proceedings of Science} - \href{https://conference.ippp.dur.ac.uk/event/1265/contributions/7291/}{Indico page}
    \item L. Giusti, M. Bresciani, M. Dalla Brida, T. Harris, D. Laudicina, M. Pepe,
    PR, \textit{Non-perturbative thermal QCD at 
    very high temperatures}, 
    42$^{\rm nd}$ International Conference on High Energy Physics, Prague, CZ - to appear in \textit{Proceedings
    of Science} - \href{https://arxiv.org/abs/2410.12626}{arXiv:2410.12626}
\end{itemize}
%%%%%%%%
\tableofcontents
\pagestyle{mystyle}

\addcontentsline{toc}{chapter}{Introduction}
\chapter*{Introduction}
%A somewhat concise ($\sim$ 3 pages) introduction about the \say{status} of 
%theoretical particle physics: SM is a successful theory, need to understand it 
%to increasing precision and in extreme conditions, need for non-perturbative 
%methods, overview of lattice QCD and its \say{philosophy} 
\pagestyle{mystyle}
The experimental discovery of the Higgs Boson at 
CERN's Large Hadron Collider in 2012 \cite{HiggsCMS, HiggsATLAS} marked the 
Standard Model of Particle Physics (SM) as the complete and self-consistent 
Quantum Field Theory describing Strong and Electro-weak interactions.
The Higgs mass and its couplings were, up to then, the only remaining free parameters 
of the SM Lagrangian, and the measurement of its value rendered the theory fully 
predictive with regards to the outcomes of succeeding experiments.\\ 
New and existing measurements are used to assess the consistency of the 
overconstrained parameters of the Standard Model, and
aside from a restriced number of cases which are still actively being 
scrutinized by the scientific community, little to no significant tension has 
been observed between experiment and theory to their current respective level of
precision. In this regard, The Standard Model is a remarkably successful theory
capable of explaining a plethora of results from accelerator and
non-accelerator based experiments. It is not uncommon for physicists and 
philosophers of science to regard it as one of mankind's most important 
scientific achievements - if not the most important \textit{tout court}.\\
The Standard Model does not, however, provide a description of
all known physical phenomena. Most notably, the gravitational
interaction is not included in the Standard Model, and it is instead described 
by Einstein's General Theory of Relativity (GR), in and of itself a formidably 
predictive and successful theory. GR differs substantially from the SM in its
formulation, specifically due to the fact that it is a \textit{classical} field 
theory. Attempts to promote General Relativity to a quantum field theory based
on standard methods fail by construction - it is a \say{non-renormalizable}
field theory, and it thus can not be made predictive when quantum corrections
are taken into account. If this inconsistency were to be solved, it would 
possibly be done in a formally new paradigm, different from QFT, from which
Quantum Field Theories (and in particular the Standard Model) would emerge
as effective low-energy descriptions. Perhaps the most well-known approach in this
direction is represented by supersymmetric string theory.\\
The experimentally determined properties of Neutrinos also elude their description
within the SM: while the observation of neutrino oscillations implies
that at least one neutrino flavor has non-zero mass, the SM describes neutrinos
as massless particles. Experimental bounds on neutrino masses pose them as orders
of magnitude lighter than other SM particles, thus the existance and the smallness
of neutrino masses directly points at the existence of beyond-standard model
(BSM) phenomena. While many SM extension have been proposed to account for
neutrino masses, experimental verification of such models remains elusive.\\
Gravitational interactions also play a key role in a class of astronomical
observations which strongly suggest that the Universe contains - and is in fact 
dominated by - a form of matter which can not be described by the Standard Model 
particle content, and is therefore referred to as Dark Matter (see \cite{DMrev}
for a recent review).\\
Another observation that does not have a satisfactory explanation within the
Standard Model is known as the Baryon asymmetry or Matter-Antimatter asymmetry
problem \cite{BaryonAsy_rev}, which essentially refers to the fact that 
ordinary (baryonic) matter is seen to hugely prevail over antimatter in the 
Universe. While the Standard Model allows for processes that could in principle
produce an overdensity of baryons \cite{EWBar_rev}, the predicted rate of 
such processes is not nearly enough to account for the observed baryon asymmetry.\\
These observations thus strongly suggest that the Standard Model needs to be at 
least extended. 
To validate any proposed solution to the above mentioned issues based on 
Sandard Model extension, a precise theoretical understanding of SM predictions is 
necessary. Quantum Field Theories like the Standard Model find their most sound
definition within the Path Integral formalism, in which the system's partition
function and correlation functions are expressed as functional integrals over 
the space of all possible field configurations. These integrals - which encode
the physical predictions of the theory, such as hadron masses, cross-sections 
and decay rates - are in most cases analytically untractable.  


\pagenumbering{arabic}

%\pagestyle{mystyle}
\chapter{Name of the chapter}

Text here


\chapter{Quantum Chromo-Dynamics on the lattice}
\section{Lattice QCD at $T=0$}
Definition of QCD on a finite lattice: Wilson gauge action, Wilson fermions,
doublers + Nielsen-Ninomiya, symmetries of the lattice/continuum actions, 
path integral, Monte Carlo and Metropolis algorithm,
correlation functions, spectrum, systematic effects (discretization, finite volume),
scale setting, renormalization, continuum limit\\
\section{Lattice QCD at $T\neq 0$}
Usual discussion


\chapter{Screening masses}
\section{Definition}
Definition of screening masses, screening correlators, relation with 0-temp.
mass.
\section{Analytic results}
Review of analytically known results (EFT, mesons, baryons)
\section{Lattice results}
Review of lattice determinations (mesons, baryons)
\subsection{JHEP '22}
Dedicated section on meson paper of '22: non-perturbative high-temperature 
strategy, comparison with PT, splitting\\
\section{Aim of the thesis}
Conclude chapter with the \alert{aim of the thesis:} compute baryonic screening
masses from the lattice at high T, and at NLO in EFT\\
Tackle exponential problem in non-static mesons, detail spin-splitting


\chapter{Original contributions}

Detail the original work I carried during Ph. D.
\section{Baryonic screening masses}
\subsection{NLO EFT}
Describe \cite{BaryonNlo}
\subsection{From the lattice}
Describe \cite{BaryonLattice} 
\section{Non-static mesonic screening masses}
Describe SNR problem, detail random sources implementation, determine screening
spectrum
\section{Hyperfine splitting in static mesons}
\subsection{At order $g^4$ in EFT perturbation theory}
\subsection{With high-precision data from lattice simulations}


\addcontentsline{toc}{chapter}{Conclusions and Outlook}
\include{conclusions/conclusions}

\appendix
\chapter{Notation and conventions}\label{app:A}


%%%% BIBLIOGRAPHY %%%%%
\addcontentsline{toc}{chapter}{Bibliography}
\printbibliography

\addcontentsline{toc}{chapter}{Acknowledgements}
\include{thankyou/thankyou}
\end{document}

